\documentclass[10pt,aspectratio=43,mathserif,table]{beamer} 
%设置为 Beamer 文档类型,设置字体为 10pt,长宽比为16:9,数学字体为 serif 风格
\batchmode

\usepackage{graphicx}
\usepackage{animate}
\usepackage{hyperref}

%导入一些用到的宏包
\usepackage{amsmath,bm,amsfonts,amssymb,enumerate,epsfig,bbm,calc,color,ifthen,capt-of,multimedia}
\usepackage{xeCJK} %导入中文包
\setCJKmainfont{SimHei} %字体采用黑体  Microsoft YaHei
\usetheme{Berlin} %主题
\usecolortheme{sustech} %主题颜色

\usepackage[ruled,linesnumbered]{algorithm2e}

\usepackage{fancybox}
\usepackage{xcolor}
\usepackage{times}
\usepackage{listings}

\usepackage{booktabs}
\usepackage{colortbl}

\newcommand{\Console}{Console}
\lstset{ %
	backgroundcolor=\color{white},   % choose the background color
	basicstyle=\footnotesize\rmfamily,     % size of fonts used for the code
	columns=fullflexible,
	breaklines=true,                 % automatic line breaking only at whitespace
	captionpos=b,                    % sets the caption-position to bottom
	tabsize=4,
	commentstyle=\color{mygreen},    % comment style
	escapeinside={\%*}{*)},          % if you want to add LaTeX within your code
	keywordstyle=\color{blue},       % keyword style
	stringstyle=\color{mymauve}\ttfamily,     % string literal style
	numbers=left, 
	%	frame=single,
	rulesepcolor=\color{red!20!green!20!blue!20},
	% identifierstyle=\color{red},
	language=c
}

\setsansfont{Microsoft YaHei}
\setmainfont{Microsoft YaHei}

\definecolor{mygreen}{rgb}{0,0.6,0}
\definecolor{mymauve}{rgb}{0.58,0,0.82}
\definecolor{mygray}{gray}{.9}
\definecolor{mypink}{rgb}{.99,.91,.95}
\definecolor{mycyan}{cmyk}{.3,0,0,0}

%题目,作者,学校,日期
\title{Scala with Wechaty}
\subtitle{\fontsize{9pt}{14pt}\textbf{Scala实现Wechaty小节}}
\author{报告人: 蔡君 \newline \newline jcai@ganshane.com}
\institute{\fontsize{8pt}{14pt}https://github.com/jcai}
\date{\today}

%学校Logo
\pgfdeclareimage[height=0.5cm]{logo}{figures/scala-wechaty.png}
\logo{\pgfuseimage{logo}\hspace*{0.3cm}}

\AtBeginSection[]
{
	\begin{frame}<beamer>
	\frametitle{\textbf{目录}}
	\tableofcontents[currentsection]
\end{frame}
}
\beamerdefaultoverlayspecification{<+->}
% -----------------------------------------------------------------------------
\begin{document}
% -----------------------------------------------------------------------------

\frame{\titlepage}

\section[目录]{}   %目录
\begin{frame}{目录}
\tableofcontents
\end{frame}

% -----------------------------------------------------------------------------
\section{自我介绍}  %引言
\begin{frame}{我玩计算机,不是计算机玩我!}
\begin{columns}[T] % align columns
\begin{column}<0->{.40\textwidth}
	\begin{figure}[thpb]
		\centering
		\resizebox{1\linewidth}{!}{
			\includegraphics{figures/jcai}
		}
		%\includegraphics[scale=1.0]{figurefile}
		% \caption{SUSTech Campus}
		\label{fig:campus}
	\end{figure}
\end{column}%
\hfill%
\begin{column}<0->{.65\textwidth}
  \begin{itemize}
    \item<1-> 15年+全职软件开发经验
          \begin{itemize}
            \item<1-> 99年开始编程;
            \item<1-> 从Pascal、C到Scala、Go;
            \item<1-> 丰富J2EE开发经验;
            \item<1-> 丰富算法开发经验,早前致力于警用指纹识别技术;
            \item<1-> 擅长C/C++,Java,Scala,Ruby,Go
          \end{itemize}
    \item<2-> 目前工作
          \begin{itemize}
            \item<2-> 军工企业创始人;
            \item<2-> 生产制造型企业;
            \item<2-> 聚焦军民融合战略资源。
          \end{itemize}
  \end{itemize}
\end{column}%
\end{columns}
\end{frame}


\section{Scala}  %自我介绍

\begin{frame}{Scala简介}
  Scala很棒,真心安利大家!
\begin{block}{函数式编程和面向对象编程}
  \lstinputlisting[lastline=11,
    language=Scala,
    frame=single,
    label=scala]
  {code/s1.scala}
\end{block}
\end{frame}

\begin{frame}{Scala简介}
  兼容Java语法,同时胜过Java语法!
\begin{block}{摘自Spring-JDBC}
  \lstinputlisting[lastline=11,
    language=java,
    frame=single]
  {code/s2.java}
\end{block}
\end{frame}
\begin{frame}{Scala简介}
  兼容Java语法,同时胜过Java语法!
\begin{block}{Scala的等同语法}
  \lstinputlisting[lastline=11,
    language=scala,
    frame=single]
  {code/s2.scala}
\end{block}
\end{frame}

\begin{frame}{Scala简介}
  静态类型
  \begin{table}
    \caption{变量定义对比}
    \begin{tabular}{|c|c|c|c|}
      \hline 
      \textbf{变量类型} & \textbf{C} & \textbf{Java} & \textbf{Scala} \\ \hline
      可变变量 & int x & int x & var x:Int \\ \hline
      不可变变量 & const int x & final int x & val x:Int \\ \hline
      延迟变量& N/A & N/A & lazy val x:Int \\ \hline
    \end{tabular}
  \end{table}
\end{frame}

\begin{frame}{Scala简介}
  类型推导
  \lstinputlisting[lastline=11,
  language=Scala,
  frame=single]
  {code/s3.scala}
\end{frame}

\begin{frame}{Scala简介}
  更加简洁的语法
  \lstinputlisting[lastline=11,
  language=Scala,
  frame=single]
  {code/s4.scala}
\end{frame}

\begin{frame}{Scala简介}
  隐式转换
  \lstinputlisting[lastline=11,
  language=Scala,
  frame=single]
  {code/s5.scala}
\end{frame}

\begin{frame}{Scala简介}
  完全兼容Java库,所有Java类库都能在Scala中使用。
\end{frame}
\begin{frame}{我如何接触上scala的?}
  通过写Scala编译器插件,实现SQL/HQL/JPQL编译时刻校验!
  \lstinputlisting[lastline=11,
  language=Scala,
  frame=single]
  {code/s7.scala}
\end{frame}


\section{Scala-Wechaty}
\begin{frame}{架构}
  \begin{itemize}
    \item<1-> wechaty => wechaty的上层调用接口;
    \item<1-> wechaty-puppet => wechaty和GRPC交互中间定义层;
    \item<1-> wechaty-hostie=> 和Grpc底层交互。
  \end{itemize}
\end{frame}
\begin{frame}{已完成工作}
  \begin{itemize}
    \item<1-> 与后端GRPC稳定交互,多日持续稳定运行;
    \item<1-> 实现所有和后端GRPC交互的代码功能;
    \item<1-> 实现所有Schema定义;
    \item<1-> 前端实现了Contact和Message部分功能;
    \item<1-> 实现了一个docker版本的ding-dong-bot\\https://github.com/wechaty/scala-wechaty/packages
  \end{itemize}
\end{frame}
\begin{frame}{已完成工作}
    使用Grpcmock(https://github.com/Fadelis/grpcmock https://github.com/tomakehurst/wiremock)实现了单元测试。
  \lstinputlisting[lastline=11,
  language=Scala,
  frame=single]
  {code/s6.scala}
\end{frame}

\section{吐槽}
\begin{frame}{文档和测试太少}
  \begin{itemize}
    \item<1-> 没有地方说明如何实现?可能我没找到。
    \item<1-> token满天飞!
    \item<1-> 架构上? on("event",func) => onEvent(func)
    \item<1-> 单元测试太少!
    \item<1-> 单元测试太少!
    \item<1-> 单元测试太少!
  \end{itemize}
\end{frame}


\section{期望和建议}
\begin{frame}{期望和建议}
  \begin{itemize}
    \item<1-> 获得更多后端GRPC细节;
    \item<1-> 精简token的使用说明;
    \item<1-> 增强ts版本的单元测试;
    \item<1-> 实现基于Atom、Electron的UI,后端能够支持各类语言的实现,方便展示和测试;
    \item<1-> Schema使用中性语言描述(考虑antlr); 
    \item<1-> puppet和hostie大部分代码能够根据schema能够自动生成,甚至wechaty都能生成出来,这个是个系统工程!
  \end{itemize}
\end{frame}



\section{未来开展工作}
\begin{frame}{Future Work}  %将来可做的方向
\begin{itemize}
\item<0-> 完善剩余wechaty功能;
\item<0-> 让更多人参入进来;
\item<0-> 更加全面的单元测试;
\item<0-> 实现一个轻量级应用(优惠券助手);
\end{itemize}
\end{frame}

\begin{frame}{Thank you}
\begin{center}
\begin{minipage}{1\textwidth}
	\setbeamercolor{mybox}{fg=white, bg=black!50!blue}
 \begin{beamercolorbox}[wd=0.70\textwidth, rounded=true, shadow=true]{mybox}
\LARGE \centering Thank you for listening!  %结束语
\end{beamercolorbox}
 \end{minipage}
\end{center}
\end{frame}

\begin{frame}{Q\&A}
\begin{center}
	\begin{minipage}{1\textwidth}
		\setbeamercolor{mybox}{fg=white, bg=black!50!blue}
		\begin{beamercolorbox}[wd=0.70\textwidth, rounded=true, shadow=true]{mybox}
			\LARGE \centering  Questions?  %请求提问
    \end{beamercolorbox}
	\end{minipage}
    https://github.com/jcai/wechaty-meetup
\end{center}
\end{frame}

% -----------------------------------------------------------------------------
\end{document}
%文档结束